\documentclass[11pt]{article}

\usepackage{amsmath}
\usepackage{amssymb}

\usepackage{enumerate}
\usepackage{enumitem}

\usepackage{geometry}
\geometry{a4paper}

\usepackage{tikz}

\begin{document}

\begin{enumerate}[8.1]
\item Any language in $SPACE(f(n))$ as defined using the two-tape read-only model can be simulated with a single tape model using at most $O(n)$ space. Similarly, any language in $SPACE(f(n))$ as defined using a single tape model can be simulated with a two-tape read-only model with an improvement of a most $O(n)$ space. Thus, the complexity classes are equivalent where $f(n) \geq n$.
\item The winning strategy for $X$ is to move to the top-right position. $O$ can then move to block only either the top-centre or centre-right position. If $O$ moves to the top-centre, $X$ moves to the centre-right. If $O$ move to the centre-right, $X$ moves to the top-centre.
\item Player I has a winning strategy as follows:
  \begin{itemize}
  \item Player I begins at node 1.
  \item Player 2 chooses node 2.
  \item Player I chooses node 4. Node 3 has only one outgoing edge which connects to node 6. As node 6 has no outgoing edges, this path would guarantee a win for Player II.
  \item Player II chooses node 5.
  \item Player I chooses node 6. As no unchosen nodes remain, Player I wins.
  \end{itemize}
\item Let $L_i$ be a language decided by PSPACE turing machine $M_i$. We define languages $L_\cup &= L_i \cup L_j$, $\bar{L_i} &= \{ w \mid w \notin L_i \}$, $L_i^* &= \{ x_1x_2 \hdots x_k$ \mid k \geq 0 \text{ and each } x_i \in L_i \}$.\\\\
  We define $M_\cup$, $\bar{M_i}$ and $M_i^*$ as follows: \\\\
  $M_\cup$ = ``On input $w$
  \begin{enumerate}[label=\arabic*.]
  \item Run $M_i\langle w \rangle$. If $M_i$ accepts, $accept$.
  \item Run $M_j\langle w \rangle$. If $M_j$ accepts, $accept$.
  \item If neither $M_i\langle w \rangle$, $M_j\langle w \rangle$ accepted, $reject$.''  \\
  \end{enumerate}
  $\bar{M_i}$ = ``On input $w$
  \begin{enumerate}[label=\arabic*.]
  \item Run $M_i\langle w \rangle$. If $M_i$ accepts, $reject$, else $accept$.'' \\
  \end{enumerate}
  $M_i^*$ = ``On input $w$
  \begin{enumerate}[label=\arabic*.]
  \item If $w = \epsilon$, $accept$.
  \item For each $m$, where $1 \leq m \leq n$, $n = |w|$.
    \begin{enumerate}[label=\arabic*.]
    \item[3.] Split $w$ into $m$ pieces, such that $w = w_1w_2 \hdots w_k$.
    \item[4.] For all $i$, $1 \leq i \leq m$, run $M_i\langle w_i \rangle$. If $M_i$ rejects, go to step 2.
    \item[5.] $M_i$ has accepted for all $i$, $accept$.
    \end{enumerate}
  \item[6.] $M_i$ has rejected for all $m$, $reject$.'' \\
  \end{enumerate}
\item Construct a TM $M$ to decide $A_\text{DFA}$ When $M$ receives input $\langle A, w \rangle$, a DFA and a string, $M$ simulates $A$ on $w$ by keeping track of $A$'s current state and its current head locations, and updating them appropriately. The space required to carry out this simulation is $O(\log{n})$ because $M$ can record each of these values by storing a pointer into its input.
\item Construct a language $L$, such that $L$ is PSPACE-hard. As $L$ is PSPACE-hard, $\text{TQBF} \leq_p L$. We know $\text{SAT} \leq_p TQBF$, which gives us $\text{SAT} \leq_p L$. As SAT is NP-complete, $L$ is NP-hard, and the proof is complete.
\item Let $A_1$ and $A_2$ be languages that are decided by NL-machines $N_1$ and $N_2$. Construct three Turing machines: $N_\cup$ deciding $A_1 \cup A_2$; $N_\circ$ deciding $A_1 \circ A_2$; and $N_*$ deciding $A_1^*$. Each of these machines operates as follows. \\\\
  Machine $N_\cup$ nondeterministically branches to simulate $N_1$ or to simulate $N_2$. In either case, $N_\cup$ accepts if the simulated machine accepts. \\\\
  Machine $N_\circ$ nondeterministically selects a position on the input to divide it into two substrings. Only a pointer to that position is stored on the work tape - insufficient space is available to store the substrings themselves. Then $N_\circ$ simulates $N_1$ on the first substring, branching nondeterministically to simulate $N_1$'s nondeterminism. On any branch that reaches $N_1$'s accept state, $N_\circ$ simulates $N_2$ on the second substring. On any branch that reaches $N_2$'s accept state, $N_\circ$ accepts. \\\\
  Machine $N_*$ has a more complex algorithm, so we describe its stages. \\\\
  $N_*$ = ``On input $w$:
  \begin{enumerate}[label=\arabic*.]
    \item Initialise two input position pointers $p_1$ and $p_2$ to 0, the position immediately preceding the first input symbol.
    \item $Accept$ if no input symbols occur after $p_2$.
    \item Move $p_2$ forward to a nondeterministically selected position.
    \item Simulate $N_1$ on the substring of $w$ from the position following $p_1$ to the position at $p_2$, branching nondeterministically to simulate $N_1$'s nondeterminism.
    \item If this branch of the simulation reaches $N_1$'s accept state, copy $p_2$ to $p_1$ and go to stage 2. If $N_1$ rejects on this branch, $reject$.''
  \end{enumerate}
\item We let $s_1,s_2,\hdots,s_k$ be the starting position, and $s_1',s_2',\hdots,s_k'$ the position after the move is made. We define $\text{NIM}_s$ to be the result of performing an XOR operation on each column in $s_1,s_2,\hdots,s_k$, such that $\text{NIM}_s = s_1 \oplus s_2 \oplus \hdots \oplus s_k$. We note that $s_1,s_2,\hdots,s_k$ is balanced if and only if $\text{NIM}_k = 0$. \\\\
  Let $s_l$ denote the pile from which sticks are removed, such that $s_i = s_i'$ for all $i \neq l$ and $s_l > s_l'$. We then have
  \begin{align*}
    \text{NIM}_s' &= 0 \oplus \text{NIM}_s' \\
                  &= \text{NIM}_s \oplus \text{NIM}_s \oplus \text{NIM}_s' \\
                  &= \text{NIM}_s \oplus (s_1 \oplus s_2 \oplus \hdots \oplus s_k ) \oplus (s_1' \oplus s_2' \oplus \hdots \oplus s_k' ) \\
                  &= \text{NIM}_s \oplus (s_1 \oplus s_1') \oplus \hdots \oplus (s_k \oplus s_k') \\
                  &= \text{NIM}_s \oplus 0 \oplus \hdots \oplus (s_l \oplus s_l') \oplus \hdots \oplus 0 \\
                  &= \text{NIM}_s \oplus s_l \oplus s_l'
  \end{align*}
  Assume $\text{NIM}_s \neq 0$. Let $d$ be the most significant non-zero bit of $\text{NIM}_s$. To move to a balanced position, We choose $l$ such that $s_l \neq 0$, and set $s_l' = \text{NIM}_s \oplus s_l$. Note that we chose $d$ in a way that maintains the constraint that $s_l' < s_l$. We then have
  \begin{align*}
    \text{NIM}_s'&= \text{NIM}_s \oplus s_l \oplus s_l' \\
                 &= \text{NIM}_s \oplus s_l \oplus (\text{NIM}_s \oplus s_l) \\
                 &= (\text{NIM}_s \oplus s_l) \oplus (\text{NIM}_s \oplus s_l) \\
                 &= 0
  \end{align*}
  Assume $\text{NIM}_s = 0$. To move to an unbalanced position, choose any $l$. We then have $\text{NIM}_s' = s_l \oplus s_l'$, where $s_l \neq s_l'$.
\item Let $a_i, b_i, c_i$ be the $i$-th digits of $a,b,c$ and $r_i$ the carry generated for $c_i$. We then have
  \begin{align*}
    r_i &= \Big(c_{i-1} + \sum_{j+k=i+1} a_jb_k \Big) \bmod 2 \\
    c_i &= \Big(\sum_{m=0}^{i-2} \sum_{j+k=i-m} a_jb_k \Big) / D
  \end{align*}
  We define machine $M$, which decides $\text{MULT}\langle a\#b\#c \rangle$. \\\\
  $M =$ ``On input $a\#b\#c$:
  \begin{enumerate}
    \item[1.] Initialise $t = 0$.
    \item[2.] Let $a = a_1a_2 \hdots a_p$, $b = b_1b_2 \hdots b_q$, $c = c_1c_2 \hdots c_{p+q-1}$.
    \item[3.] For $k = 0$ to $k = p + q - 1$
    \begin{enumerate}
      \item[4.] For $j = max(1, k - p + 1)$ to $min(k, q)$.
      \begin{enumerate}
        \item[5.] Let $i = k - j + 1$.
        \item[6.] Update $t = t + (a_ib_j)$.
      \end{enumerate}
      \item[7.] If $c_k = t \bmod 2$, continue, else $reject$.
      \item[8.] Update $t = \lfloor t / 2 \rfloor$.
    \end{enumerate}
    \item[9.] If $c_{p+q} = t \bmod 2$, $accept$, else $reject$.
  \end{enumerate}
  As $M$ stores $i,j,k$ only as pointers to the input tape, they require only $O(\log n)$ of worktape space. We also note that the sum and carry can never exceed values larger than $O(\log n)$ bits, meaning that $t$ requires only $O(\log n)$ worktape space. We have then shown that $M$ can decide MULT using no more than logarithmic space, which completes the proof that $\text{MULT} \in L$.
\end{document}
