\documentclass[11pt]{article}

\usepackage{amsmath}
\usepackage{amssymb}

\usepackage{enumerate}
\usepackage{enumitem}

\usepackage{geometry}
\geometry{a4paper}

\usepackage{tikz}

\begin{document}

\begin{enumerate}[4.1]
  \item
  \begin{enumerate}
    \item Yes. $M$ accepts $0100$.
    \item No. $M$ rejects $011$.
    \item No. No input string is given.
    \item No. $R$ is not a DFA.
    \item No. $L(M)$ is not empty.
    \item Yes. By definition, $L(M) = L(M)$.
  \end{enumerate}
  \item We define the language $EQ_{DR}$ as
    \begin{align*}
      EQ_{DR} = \{ \langle A, R \rangle \mid A \text{ is a DFA}, R \text{ is a regular expression, and } L(A) = L(R) \}
    \end{align*}
    To show $EQ_{DR}$ is decidable, we construct Turing machine $M$ which decides $EQ_{DR}$. \\\\
    $M =$ ``On input $\langle A, R \rangle$
    \begin{enumerate}[label=\arabic*.]
      \item Construct DFA $B$ which recognises $R$ (see theorem 1.39).
      \item Run $EQ_{DFA}\langle A, B \rangle$ (see theorem 4.5). If $EQ_{DFA}\langle A, B \rangle$ accepts, $accept$, else $reject$.''
    \end{enumerate}
\end{enumerate}
\end{document}
