\documentclass[11pt]{article}

\usepackage{amsmath}
\usepackage{amssymb}

\usepackage{enumerate}

\usepackage{geometry}
\geometry{a4paper}

\renewcommand{\indent}{\hspace{1em}}

\begin{document}

\begin{enumerate}[7.1]
  \item
  \begin{enumerate}
    \item True.
    \item False.
    \item False.
    \item True.
    \item True.
    \item True.
  \end{enumerate}
  \item
  \begin{enumerate}
    \item False.
    \item True.
    \item True.
    \item True.
    \item False.
    \item False.
  \end{enumerate}
  \item Following Euclids algorithm:
  \begin{enumerate}
  \item Yes.
    \begin{align*}
                       & GCD(10505, 1274) & \\
      =\hspace{0.25em} & GCD(1274, 313)   & \\
      =\hspace{0.25em} & GCD(313, 22)     & \\
      =\hspace{0.25em} & GCD(22, 5)       & \\
      =\hspace{0.25em} & GCD(5, 2)        & \\
      =\hspace{0.25em} & GCD(2, 1 )       & \\
      =\hspace{0.25em} & GCD(1, 1)        & \\
      =\hspace{0.25em} & 1                & \\
    \end{align*}
  \item No.
    \begin{align*}
                       & GCD(8024, 7289) & \\
      =\hspace{0.25em} & GCD(7289, 740)  & \\
      =\hspace{0.25em} & GCD(740, 629)   & \\
      =\hspace{0.25em} & GCD(629, 111)   & \\
      =\hspace{0.25em} & GCD(111, 74)    & \\
      =\hspace{0.25em} & GCD(74, 37)     & \\
      =\hspace{0.25em} & 37.
    \end{align*}
  \end{enumerate}
  \item
  \item Let $\phi = (x \lor y) \land (x \lor \bar{y}) \land (\bar{x} \lor y) \land (\bar{x} \lor \bar{y})$. The following truth-table shows that $\phi$ is not satisfiable. \\\\
    \begin{tabular}{|c|c|c|c|c|c|c|} \hline
      x & y & $(x \lor y)$ & $(x \lor \bar{y})$ & $(\bar{x} \lor y)$ & $(\bar{x} \lor \bar{y})$ & $\phi$ \\ \hline
      0 & 0 & 0            & 1                  & 1                  & 1                        & 0      \\ \hline
      0 & 1 & 1            & 0                  & 1                  & 1                        & 0      \\ \hline
      1 & 0 & 1            & 1                  & 0                  & 1                        & 0      \\ \hline
      1 & 1 & 1            & 1                  & 1                  & 0                        & 0      \\ \hline
    \end{tabular} \\
  \item Let $\langle L_i$, $M_i \rangle$ be a polytime language and decider, such that $L_i = \{ w \mid M_i\langle w \rangle \text{ accepts} \}$, and $M_i$ always halts in polynomial time. \\\\
    Let $L_\cup = L_i \cup L_j$, $L_\circ = L_i \circ L_j$, $\bar{L_i} = \{ w \in \Sigma^* \mid w \notin L_i \}$ be languages. To show that $L_\cup, L_\circ, \bar{L_i} \in$ P, we construct respective polynomial time deciders $M_\cup, M_\circ, \bar{M_i}$. \\\\
    $M_\cup$ = ``On input $w$:
    \begin{enumerate}[\indent1.]
      \item Run $M_i\langle w \rangle$. If $M_i$ accepts, $accept$.
      \item Run $M_j\langle w \rangle$. If $M_j$ accepts, $accept$.
      \item If neither $M_i\langle w \rangle$ nor $M_j\langle w \rangle$ accepted, $reject$.''
    \end{enumerate}
    $M_\circ$ = ``On input $w$:
    \begin{enumerate}[\indent1.]
      \item For each position $k = 0$ to $|w|$, divide $w$ into substrings $w = w_1w_2$, where $w_1$ is the first $k$ symbols in $w$.
      \item \indent Run $M_i\langle w_1 \rangle$ and $M_j\langle w_2 \rangle$. If both accept, $accept$.
      \item If no $k$ exists such that $M_i\langle w_1 \rangle$ and $M_j\langle w_2 \rangle$ both $accept$, $reject$.''
    \end{enumerate}
    $\bar{M_i}$ = ``On input $w$:
    \begin{enumerate}[\indent1.]
      \item Run $M_i$ on $w$. If $M_i$ accepts $reject$. If $M_i$ rejects, $accept$.
    \end{enumerate}
\end{enumerate}

\end{document}
