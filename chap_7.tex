\documentclass[11pt]{article}

\usepackage{amsmath}
\usepackage{amssymb}

\usepackage{enumerate}
\usepackage{enumitem}

\usepackage{geometry}
\geometry{a4paper}

\renewcommand{\indent}{\hspace{1em}}

\begin{document}

\begin{enumerate}[7.1]
  \item
  \begin{enumerate}
    \item True.
    \item False.
    \item False.
    \item True.
    \item True.
    \item True.
  \end{enumerate}
  \item
  \begin{enumerate}
    \item False.
    \item True.
    \item True.
    \item True.
    \item False.
    \item False.
  \end{enumerate}
  \item Following Euclids algorithm:
  \begin{enumerate}
  \item Yes.
    \begin{align*}
                       & GCD(10505, 1274) & \\
      =\hspace{0.25em} & GCD(1274, 313)   & \\
      =\hspace{0.25em} & GCD(313, 22)     & \\
      =\hspace{0.25em} & GCD(22, 5)       & \\
      =\hspace{0.25em} & GCD(5, 2)        & \\
      =\hspace{0.25em} & GCD(2, 1 )       & \\
      =\hspace{0.25em} & GCD(1, 1)        & \\
      =\hspace{0.25em} & 1                & \\
    \end{align*}
  \item No.
    \begin{align*}
                       & GCD(8024, 7289) & \\
      =\hspace{0.25em} & GCD(7289, 740)  & \\
      =\hspace{0.25em} & GCD(740, 629)   & \\
      =\hspace{0.25em} & GCD(629, 111)   & \\
      =\hspace{0.25em} & GCD(111, 74)    & \\
      =\hspace{0.25em} & GCD(74, 37)     & \\
      =\hspace{0.25em} & 37.
    \end{align*}
  \end{enumerate}
  \item
  \item Let $\phi = (x \lor y) \land (x \lor \bar{y}) \land (\bar{x} \lor y) \land (\bar{x} \lor \bar{y})$. The following truth-table shows that $\phi$ is not satisfiable. \\\\
    \begin{tabular}{|c|c|c|c|c|c|c|} \hline
      x & y & $(x \lor y)$ & $(x \lor \bar{y})$ & $(\bar{x} \lor y)$ & $(\bar{x} \lor \bar{y})$ & $\phi$ \\ \hline
      0 & 0 & 0            & 1                  & 1                  & 1                        & 0      \\ \hline
      0 & 1 & 1            & 0                  & 1                  & 1                        & 0      \\ \hline
      1 & 0 & 1            & 1                  & 0                  & 1                        & 0      \\ \hline
      1 & 1 & 1            & 1                  & 1                  & 0                        & 0      \\ \hline
    \end{tabular} \\
  \item Let $\langle L_i$, $M_i \rangle$ be a polytime language and decider, such that $L_i = \{ w \mid M_i\langle w \rangle \text{ accepts} \}$, and $M_i$ always halts in polynomial time. \\\\
    Let $L_\cup = L_i \cup L_j$, $L_\circ = L_i \circ L_j$, $\bar{L_i} = \{ w \in \Sigma^* \mid w \notin L_i \}$ be languages. To show that $L_\cup, L_\circ, \bar{L_i} \in$ P, we construct respective polynomial time deciders $M_\cup, M_\circ, \bar{M_i}$. \\\\
    $M_\cup$ = ``On input $w$:
    \begin{enumerate}[label=\arabic*.]
      \item Run $M_i\langle w \rangle$. If $M_i$ accepts, $accept$.
      \item Run $M_j\langle w \rangle$. If $M_j$ accepts, $accept$.
      \item If neither $M_i\langle w \rangle$ nor $M_j\langle w \rangle$ accepted, $reject$.''
    \end{enumerate}
    $M_\circ$ = ``On input $w$:
    \begin{enumerate}[label=\arabic*.]
      \item For each position $k = 0$ to $|w|$, divide $w$ into substrings $w = w_1w_2$, where $w_1$ is the first $k$ symbols in $w$.
      \item \indent Run $M_i\langle w_1 \rangle$ and $M_j\langle w_2 \rangle$. If both accept, $accept$.
      \item If no $k$ exists such that $M_i\langle w_1 \rangle$ and $M_j\langle w_2 \rangle$ both $accept$, $reject$.''
    \end{enumerate}
    $\bar{M_i}$ = ``On input $w$:
    \begin{enumerate}[label=\arabic*.]
      \item Run $M_i$ on $w$. If $M_i$ accepts $reject$. If $M_i$ rejects, $accept$.
    \end{enumerate}
  \item Let $\langle L_i, N_i \rangle$ be a language and non-deterministic Turing machine, such that $N_i$ decides $L_i$ in polynomial time. \\\\
    Let $L_\cup = L_i \cup L_j$, $L_\circ = L_i \circ L_j$ be languages. To show that $L_\cup, L_\circ \in$ P, we construct respective polynomial time non-deterministic deciders. \\\\
    $N_\cup$ = ``On input $w$:
    \begin{enumerate}[label=\arabic*.]
      \item Non-deterministically branch to simulate both $N_i\langle w \rangle$ and $N_j\langle w \rangle$.
      \item If either branch accepts, $accept$, else $reject$.''
    \end{enumerate}
    $N_\circ$ = ``On input $w$:
    \begin{enumerate}
      \item[1.] Non-determinsitically branch for each position $k = 0$ to $|w|$, with each branch dividing $w$ into substrings $w = w_1w_2$, where $w_1$ is the first $k$ symbols in $w$.
      \begin{enumerate}
        \item[2.] For each branch, run $N_i\langle w_1 \rangle$ and $N_j\langle w_2 \rangle$. If both accept, $accept$.
      \end{enumerate}
      \item[3.] If no branch accepts, $reject$.''
    \end{enumerate}
  \item To show that $\text{CONNECTED} \in P$, we need to show that $\text{CONNECTED} \in \text{TIME}(n^k)$, where $k$ is some constant. As $M$ decides CONNECTED, we only need to prove that $M$ runs in $O(n^k)$.
    \begin{itemize}
      \item In step 1, a node is selected and marked. This is done in constant time.
      \item In stages 2 and 3, each node in $G$ is scanned and marked if it is a neighbour of a marked node. This is repeated until no new nodes are marked. Assuming we do not visit already marked nodes, this has a worst-case run-time of $n \sum_{i=1}^{n-1} i = \frac{n^2(n - 1)}{2}$ steps.
      \item In stage 4, each node of $G$ is scanned to determine if it is marked. This takes $n$ steps.
    \end{itemize}
    Combining steps 2,3 and 4, we have a run time of $\frac{n^2(n-1)}{2} + n = \frac{n^3 + n^2 + 2n}{2} = O(n^3)$. This is clearly in P, and the proof is complete.
  \item To show that $\text{TRIANGLE} \in \text{P}$, we need to show that $\text{TRIANGLE} \in \text{TIME}(n^k)$, where $k$ is some constant. We define Turing machine $M$, which operates as follows.\\\\
    $M =$ ``On input $\langle G \rangle$:
    \begin{enumerate}
      \item[1.] For each edge $u, v$ in $G$, do
        \begin{enumerate}
        \item[2.] For each vertex $w$, do
          \begin{enumerate}
            \item[3.] If $v, w$ is an edge and $w, u$ is an edge, $accept$
          \end{enumerate}
        \end{enumerate}
      \item[4.] If no edge, vertex combination has accepted, $reject$.
    \end{enumerate}
    $M$ runs in $O(|V||E|) = O(n^3)$. This is clearly in P, and the proof is complete.''
  \item[26.] Assume an arbitrary upper configuration of size $k > 3$. Ignoring the lower configuration, there must then be at least two overlapping windows consisting of adjacent tape symbols $abc$, where $cell[i,j-1] = a$, $cell[i,j] = b$, $cell[i,j+1] = c$. We note that any cell adjacent to the tape head may be updated in the lower configuration. As both windows contain only tape symbols, each window cell is potentially adjacent to the head (with the exception of the case in which either window is at the beginning or end of the tape), meaning neither window can verify any corresponding cells in the lower configuration. As $cell[i,j]$ is not covered by any other windows, $cell[i+1,j]$ will never be verified, and the proof is complete.
  \item[49.] To prove that $\text{DOUBLE-SAT}\langle \phi \rangle$ is NP-complete, we first show that $\text{DOUBLE-SAT} \in \text{NP}$.\\\\
    Let $N$ be a non-deterministic Turing machine which decides $\text{DOUBLE-SAT}\langle \phi \rangle$. $N$ decides DOUBLE-SAT in the same way that $\text{SAT}$ is solved, with the small modification that $N$ guesses two assignments for each branch, accepting only if both assignments satisfy $\phi$.\\\\
    To complete the proof, we give a reduction from $\text{SAT}$ to $\text{DOUBLE-SAT}$. Let $N'$ be a non-deterministic Turing machine which decides $\text{SAT}\langle \phi \rangle$.\\\\
    $M$ = ``On input $\langle \phi \rangle$
    \begin{enumerate}[label=\arabic*.]
      \item Introduce dummy variable $y$, such that $\phi' = \phi \land (y \lor \bar{y})$.
      \item If $\text{DOUBLE-SAT}\langle \phi' \rangle$ accepts, $accept$, else, $reject$.''
    \end{enumerate}
    We then have both $\text{DOUBLE-SAT} \in \text{NP}$ and $\text{SAT} \leq_p \text{DOUBLE-SAT}$, and the proof is complete.
\end{enumerate}
\end{document}
