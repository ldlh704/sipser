\documentclass[11pt]{article}

\usepackage{amsmath}
\usepackage{amssymb}

\usepackage{enumerate}

\usepackage{geometry}
\geometry{a4paper}

\begin{document}

\begin{enumerate}[0.1]
  \item
  \begin{enumerate}
    \item Odd natural numbers.
    \item Even integers.
    \item Even natural numbers.
    \item Even natural numbers and positive multiples of three.
    \item Palendromes.
    \item Empty set.
  \end{enumerate}
  \item
  \begin{enumerate}
    \item $\{1,10,100\}$
    \item $\{5,6,7,8,...\}$, $\{n \in \mathbb{Z} \mid n > 5\}$
    \item $\{0,1,2,3,4,5\}$, $\{n \in \mathbb{N} \mid n < 5\}$
    \item $\{\texttt{aba}\}$, $\{w \mid w \text{ is the string } \texttt{aba}\}$
    \item $\{\epsilon\}$, $\{w \mid w \text{ is the empty string}\}$
    \item $\emptyset$
  \end{enumerate}
  \item
  \begin{enumerate}
    \item No.
    \item Yes.
    \item $A$
    \item $B$
    \item $\{(x,x),(x,y),(y,x),(y,y),(z,x),(z,y)\}$
    \item $\{\emptyset,\{x\},\{y\},\{x,y\}\}$
  \end{enumerate}
  \item
  \begin{enumerate}
    \item $a \times b$
    \item $\sum_{k=0}^n \binom{n}{k} = 2^n$
  \end{enumerate}
  \item
  \begin{enumerate}
    \item $7$
    \item $\{6,7\}, \{1,2,3,4,5\}$
    \item $6$
    \item $\{6,7,8,9,10\}, \{1,2,3,4,5\} \times \{6,7,8,9,10\}$
    \item $g(4, f(4)) = g(4, 7) = 8$
  \end{enumerate}
  \item
  \begin{enumerate}
    \item $\approx$
    \item $\leq$
    \item $isAdjacent$
  \end{enumerate}
  \item $deg(1)=3$, $deg(2)=3$, $deg(3)=2$, $deg(4)=2$
  \item 
  \def \vertex [#1]#2 {\draw[fill=#1] #2 circle (6pt);}
  \def \label [#1]#2 {\node at #1 {#2};}
  \def \edge #1#2 {\draw #1 -- #2;}
  \def \pathEdge #1#2 {\draw[thick] #1 -- #2;}
  \begin{tikzpicture}[baseline]
    % vertices
    \vertex[black]{(0,0)}
    \vertex[black]{(4,0)}
    \vertex[black]{(0,-4)}
    \vertex[black]{(4,-4)}
    % labels
    \label[(-0.5,0)]{1}
    \label[(4.5,0)]{2}
    \label[(-0.5,-4)]{3}
    \label[(4.5,-4)]{4}
    % edges
    \edge{(0,0)}{(4,0)}
    \edge{(4,0)}{(0,-4)}
    \edge{(0,0)}{(0,-4)}
    \edge{(4,0)}{(4,-4)}
    \edge{(0,0)}{(4,-4)}
    % path
     \pathEdge{(0,-4)}{(4,0)}
     \pathEdge{(4,0)}{(4,-4)} 
  \end{tikzpicture}
  \item $G = (\{1,2,3,4,5,6\}, \{(1,4),(1,5),(1,6),(2,4),(2,5),(2,6),(3,4),(3,5),(3,6)\})$
  \item Take a graph $G = (V,E)$, where $|V| \geq 2$. To prove the statement true, we need to show it is not possible to construct $G$ without two vertices having the same degree. \par As each edge connects a pair of vertices, the set of all possible degrees for a given vertex $v \in G$ is  $D = \{0, 1, \dots, (|V|-1)\}$. For the statement to be true, no two vertices may have the same degree, implying a bijection between $D$ and $V$. Thus, $V$ must include vertices $v, u$ where $D(v) = (|V|-1)$, $D(u) = 0$. This is a contradiction, which completes the proof. 
  \item The proof only establishes that horses in $H_1$ and $H_2$ have the same colour when $|$H_1$| = |$H_2| = 1$. It does not establish that horses in $H_1 \cup H_2$ are the same colour.
  \item 
  \begin{enumerate}
    \item Prove $S(n) = 1 + 2 + \hdots + n = \frac{1}{2}n(n+1)$.
    \begin{itemize}
      \item Base case:
      \begin{flalign*}
        S(1) &= \frac{1}{2}(1)(2) = 1 &&
      \end{flalign*}
      \item Inductive case:
      \begin{flalign*}
        S(n) &= S(n-1) + n \\
                &= \frac{1}{2}(n-1)(n - 1 + 1) + n \\
                &= \frac{1}{2}(n^2 + n) \\
                & = \frac{1}{2}n(n+1) &&
      \end{flalign*} 
    \end{itemize}
    \item Prove $C(n) = 1^3 + 2^3 + \hdots + n^3 = \frac{1}{4}n^2(n + 1)^2$.
    \begin{itemize}
      \item Base case:
      \begin{flalign*}
        C(1) &= \frac{1}{4}(1)^2(2)^2 = 1 &&
      \end{flalign*}
      \item Inductive case:
      \begin{flalign*}
        C(n) &= C(n-1) + n^3 \\
                &= \frac{1}{4}(n-1)^2n^2 + n^3 \\
                &= \frac{1}{4}(n^4-2n^3+n^2) + n^3 \\
                &= \frac{1}{4}(n^4+2n^3+n^2) \\
                &= \frac{1}{4}n^2(n + 1)^2 &&
      \end{flalign*}       
    \end{itemize}
  \end{enumerate}
  \item Division by $a-b = 0$.
  \item Solution in book.
  \item Solution in book.
\end{enumerate}

\end{document}
